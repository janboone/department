% Intended LaTeX compiler: pdflatex
\documentclass[presentation]{beamer}
\usepackage[utf8]{inputenc}
\usepackage[T1]{fontenc}
\usepackage{graphicx}
\usepackage{longtable}
\usepackage{wrapfig}
\usepackage{rotating}
\usepackage[normalem]{ulem}
\usepackage{amsmath}
\usepackage{amssymb}
\usepackage{capt-of}
\usepackage{hyperref}
\usepackage{listings}
\usepackage{minted}
\usetheme[height=20pt]{Montpellier}
\author{Jan Boone}
\date{Sept. 29, 2023}
\title{Discussion of  "Patient cost-sharing and risk solidarity in health insurance"}
\title[Patient cost-sharing]{Discussion of "Patient cost-sharing and risk solidarity in health insurance"}
\hypersetup{
 pdfauthor={Jan Boone},
 pdftitle={Discussion of  "Patient cost-sharing and risk solidarity in health insurance"},
 pdfkeywords={},
 pdfsubject={},
 pdfcreator={Emacs 29.1 (Org mode 9.7-pre)}, 
 pdflang={English}}
\begin{document}

\maketitle

\begin{frame}[label={sec:orgd47a23a}]{Summary: approach}
\begin{itemize}
\item paper analyzes whether higher levels of cost sharing increase risk solidarity and welfare for people with high health risks
\item risk classes defined on the basis of last year's expenditure
\item model follows each insured per month over a year
\item state variable: remaining deductible
\item with some probability patient needs care (conditional on last month's care)
\item there is a "true" need \(\lambda\) and optional need \(\omega\) if care is free
\item insured take into account that spending now makes care "cheaper" in the future
\item estimated model is used to simulate/predict outcomes for different cost sharing schemes
\end{itemize}
\end{frame}

\begin{frame}[label={sec:orgd0a4c32}]{Summary: findings}
\begin{itemize}
\item higher cost sharing than we currently have in the Netherlands reduces expenditure, premium and increases welfare, also for the high risks
\item main effect is consumption reduction among low risk types
\end{itemize}
\end{frame}

\begin{frame}[label={sec:org6f33641}]{Summary: disclaimer}
\begin{itemize}
\item I am biased: we tend to find smaller effects
\item read the paper with the view: what explains the different outcomes in the approaches
\end{itemize}
\end{frame}


\begin{frame}[label={sec:org62da66d}]{what I like about the paper}
\begin{itemize}
\item focus on risk solidarity
\item explicitly distinguish different risk classes
\item dynamic optimization problem taking remaining deductible into account
\begin{itemize}
\item solution by backward induction
\end{itemize}
\item model choices based on 2 principles
\item intuitive figures to illustrate model fit
\item model can be used to simulate outcomes for cost sharing schemes that were never implemented before
\end{itemize}
\end{frame}


\begin{frame}[label={sec:org02b6434}]{things I do not quite get: analysis}
\begin{itemize}
\item paper suggests 0-1 decision on \(\omega\), but there is a margin here as well:
\begin{itemize}
\item e.g. \(\omega=80\) remaining deductible equals 79 vs 1 euro remaining
\end{itemize}
\item expenditure is defined as "aggregate payments for claims in each month, based on the date claims were initiated."
\begin{itemize}
\item does this over-estimate \(p^1\)?
\item if I get, drugs, in one month and physiotherapy in the next as part of the same treatment for tennis elbow?
\end{itemize}
\item moral hazard is modeled as additive
\begin{itemize}
\item seems counterintuitive: the more you (really) need, the more you can add
\item multiplicative seems sensible?
\end{itemize}
\end{itemize}
\end{frame}


\begin{frame}[label={sec:org8bb453d}]{things I do not quite get: effect size}
\begin{itemize}
\item table 5: Q1 halves expenditure from D=0 to D =350
\begin{itemize}
\item halves it again to D=500
\item for higher quartiles effects are smaller
\item our findings suggest: Q1 hardly goes to the doctor and if they go it is something serious
\end{itemize}
\begin{itemize}
\item oop expenditure falls as \(D\) increases from 350 to 500 euro:
\begin{itemize}
\item huge behavioral effect
\item for many people necessary expenditure is already above 500 euro?
\end{itemize}
\end{itemize}
\item 75\% coinsurance with 350 max. leads to higher oop than \(D=350\)?
\begin{itemize}
\item in each state of the world you pay less
\end{itemize}
\end{itemize}
\end{frame}


\begin{frame}[label={sec:orgf893629}]{things I do not quite get: interpretation}
\begin{center}
\begin{tabular}{lrrrr}
risk class & Q1 & Q2 & Q3 & Q4\\[0pt]
\hline
\(E(c \mid c>0\)) & 177 & 164 & 184 & 259\\[0pt]
\(\omega\) & 87 & 71 & 73 & 17\\[0pt]
\end{tabular}
\end{center}

\begin{itemize}
\item once I fill my deductible, every following month (with probability \(p^{0,1}\))  I will spend \(\omega\)
\item small increase in \(D\) that prevents healthy people from outspending \(D\) early in the year has very big effect
\item \(\delta\) is higher for higher risk scores
\begin{itemize}
\item but higher probability of dying?
\end{itemize}
\end{itemize}
\end{frame}


\begin{frame}[label={sec:org062682c}]{things I do not quite get: robustness}
\begin{itemize}
\item risk solidarity is more important than risk aversion
\begin{itemize}
\item absurd level of risk aversion is needed to overturn results
\item but the Dutch\ldots{}
\end{itemize}
\item liquidity constraints/behavioral hazard
\begin{itemize}
\item model does not allow for \(c < \lambda\)
\begin{itemize}
\item hard to interpret arguments about unmet needs
\item model only captures not visiting a doctor when one has needs
\item not the fact that visit was made but under-spending \(c < \lambda\).
\end{itemize}
\end{itemize}
\end{itemize}
\end{frame}

\begin{frame}[label={sec:orge11f7b3}]{things I do not quite get: more broadly}
\begin{itemize}
\item paper basically argues that everybody gains if we increase the deductible level; even high risks:
\begin{itemize}
\item so why do we have this discussion in the Netherlands?
\item which part of the analysis do people overlook?
\end{itemize}
\item paper argues that risk aversion or liquidity constraints cannot overturn these results
\begin{itemize}
\item but these arguments come back in the policy debate?
\end{itemize}
\end{itemize}
\end{frame}
\end{document}