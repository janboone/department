% Intended LaTeX compiler: pdflatex
\documentclass[presentation]{beamer}
\usepackage[utf8]{inputenc}
\usepackage[T1]{fontenc}
\usepackage{graphicx}
\usepackage{longtable}
\usepackage{wrapfig}
\usepackage{rotating}
\usepackage[normalem]{ulem}
\usepackage{amsmath}
\usepackage{amssymb}
\usepackage{capt-of}
\usepackage{hyperref}
\usetheme{default}
\author{Jan Boone}
\date{Econ HalfDay}
\title{Teaching economics students programming?}
\hypersetup{
 pdfauthor={Jan Boone},
 pdftitle={Teaching economics students programming?},
 pdfkeywords={},
 pdfsubject={},
 pdfcreator={Emacs 27.2 (Org mode 9.6)}, 
 pdflang={English}}
\begin{document}

\maketitle

\begin{frame}[label={sec:org31c42bf}]{Why should economists be able to program?}
\begin{itemize}
\item the more skills the better\ldots{}
\item but we only have limited time to teach (BSc) students
\item trade off: teaching less math
\end{itemize}
\end{frame}


\begin{frame}[label={sec:orgbf4996a}]{math vs programming}
\begin{itemize}
\item both teach abstract thinking
\item more ``realistic'' models
\item generating your own data
\item useful skills after graduation
\end{itemize}
\end{frame}


\begin{frame}[label={sec:org31bd102}]{abstract thinking}
\begin{itemize}
\item one of the advantages of math is that it teaches students to think in a formal and abstract way
\item but this is also true for programming:
\begin{itemize}
\item program consumers with different utility functions
\item derive their demand for a product with different income and price levels
\item changing the price, moves along the demand curve
\item changing income shifts the demand curve
\end{itemize}
\end{itemize}
\end{frame}

\begin{frame}[label={sec:org26611aa}]{although I am a theorist\ldots{}}
\begin{itemize}
\item after teaching BSc and MSc students economic theory courses for 20 years
\item I am not sure they ``get it''
\item many view a model as a complicated way to state a simple intuition
\item why not do the intuition rightaway?
\end{itemize}
\end{frame}

\begin{frame}[label={sec:org60afc0e}]{more realistic models}
\begin{itemize}
\item Rethinking Economics NL: Neoclassical models are ``too simplistic''
\item models are meant to be simple
\item but in our BSc programs, models are also simple because otherwise students cannot solve them
\begin{itemize}
\item if students' math skills only allow for solving symmetric models, you cannot model inequality
\end{itemize}
\item complicated models can be solved using programming
\item to understand robustness/sensitivity w.r.t. assumptions: solve the model 1000 times
\end{itemize}
\end{frame}


\begin{frame}[label={sec:org72f2f30}]{generating your own data}
\begin{itemize}
\item students tend to see a dichotomy:
\begin{itemize}
\item you either do theory
\item or estimate a model
\end{itemize}
\item to teach them that this is a ``continuum'':
\begin{itemize}
\item program a (theory) model
\item generate data from this model
\item estimate the equation you are interested in, on this data
\end{itemize}
\end{itemize}
\end{frame}

\begin{frame}[label={sec:org0fc1bd5}]{also works for econometrics}
\begin{itemize}
\item specify a data generating process
\item generate a sample
\item estimate a slope parameter
\item repeat this 1000 times:
\begin{itemize}
\item a slope has a distribution\ldots{}
\end{itemize}
\item see how instrumental variables work; causality etc.
\end{itemize}
\end{frame}

\begin{frame}[label={sec:org52e97e5}]{useful skills after graduation}
\begin{itemize}
\item many students, years after graduation, have confided that they had not maximized a function in the past 10 years
\item math skills seem useful during the years of BSc and MSc economics, but not afterwards
\item this could be different for programming:
\begin{itemize}
\item scraping websites for data
\item interactive graphs for presentations
\item neural networks and datascience
\item general computer skills: e.g. folders
\end{itemize}
\end{itemize}
\end{frame}
\end{document}
